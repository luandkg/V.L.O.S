
\documentclass[article,12pt,oneside,a4paper,english,brazil,sumario=tradicional]{abntex2}		
% Pacotes usados
\usepackage{times}%Usa a fonte Latin Modern
\usepackage[T1]{fontenc}%Selecao de codigos de fonte.
\usepackage[utf8]{inputenc}%Codificacao do documento
\usepackage{indentfirst}%Indenta o primeiro parágrafo de cada seção.
\usepackage{nomencl}%Lista de simbolos
\usepackage{color}%Controle das cores
\usepackage{graphicx}%Inclusão de gráficos
\usepackage{microtype}%Para melhorias de justificação
\usepackage{lipsum}%Para geração de dummy text
\usepackage[abnt-emphasize=bf,abnt-and-type=e,alf]{abntex2cite}%Citações ABNT
\usepackage{mathptmx}
%\usepackage[bottom=2cm,top=3cm,left=3cm,right=2cm]{geometry}

% Configuracoes do documento
\graphicspath{{./Figuras/}}%Images na pasta "Figuras"
\setsecheadstyle{\bfseries \normalsize \uppercase}
\setsubsecheadstyle{\normalsize \uppercase}
\setsubsubsecheadstyle{\bfseries \normalsize}
\setlrmarginsandblock{3cm}{2cm}{*}%Margens esquerda-direita
\setulmarginsandblock{3cm}{2cm}{*}%Margens cima-baixo
\checkandfixthelayout
\setlength{\parindent}{1.25cm}%paragrafo
\OnehalfSpacing%espacamento de 1,5
\setlength{\ABNTEXcitacaorecuo}{4cm}%recuo citacao direta +3

\begin{document}
\selectlanguage{brazil} % Seleciona o idioma do documento
\frenchspacing % Retira espaço extra obsoleto entre as frases.

\begin{center}
%TITULO
	\uppercase{\bfseries{Sistemas Operacionais - Projeto VLOS}}
	\vspace{12pt}
\end{center}

\begin{flushright}
%AUTOR - Pode-se contar com infinitos autores   :)
	Luan Freitas\footnote{Luan Freitas, 170003191@aluno.unb.br}
	\\
	Vinícius Martins\footnote{Vinicius Martins, 170157962@aluno.unb.br}
	\vspace{12pt}
\end{flushright}

\begin{footnotesize}
\SingleSpacing
\noindent
\small{\textbf{Resumo:}}
\noindent
\small
Implementação do projeto VLOS - Vinícius e Luan Operating System para a disciplina de Introdução à Sistemas Operacionais, com gerência de memória, gerência de processos, gerência de  recursos e gerência de arquivos.

\noindent
%PALAVRAS-CHAVE} 
\textbf{Palavras-chave}: VLOS, Sistema Operacional, Processos e Memória.
\end{footnotesize}

\textual
\pagestyle{simple}
\aliaspagestyle{chapter}{simple}


\section{Introdu\c c\~ao}
\label{secIntroducao}
\normalsize
% TEXTO DA INTRODUCAO
A ideia central é a implementação de um Pseudo Sistema Operacional com gerência de memória, gerência de processos e gerência de recursos, implementando as técnicas de paginação para alocação de memória e escalonamento Robin Round \cite {maziero2019} para gerência e fluxo de processos de usuário,com a técnica de First Fit \cite {tanebaum2010} para processos do Kernel e sistema de prioridade em múltiplas filas para processos de usuários;

O projeto utiliza dois arquivos de entrada de dados para carregar o fluxo de tarefas a serem executadas pelo sistema operacional.

\section{Desenvolvimento}
% TEXTO DO DESENVOLVIMENTO
A primeira decisão foi a linguagem de programação utilizada pelo projeto, no caso JAVA, o principal motivo da escolha foi porque ambos os implementadores do projeto possuem um bom domínio da linguagem e conhecimento de técnicas de padrões de projeto utilizadas em JAVA.

A segunda decisão foi se seria necessário a requisição de alguma biblioteca de terceiros para o projeto, no caso ficou de acordo que nao seria realizada nenhuma dependência de bibliotecas *.jars para o projeto.

O projeto foi totalmente desenvolvido de forma versionada utilizando ferramenta GIT e a plataforma de compartilhamento GITHUB.
 
\vspace{24pt} %dois espacos de 12
\begin{citacao}
"A fim de projetar um algoritmo de escalonamento, é necessário ter alguma ideia do que um bom algoritmo deve fazer. Certas metas dependem do ambiente (em lote, interativo ou de tempo real), mas algumas são desejáveis em todos os casos. \cite[p.~106]{tanebaum2010}. 
\end{citacao}
\vspace{24pt} %dois espacos de 12

O projeto foi divido em pacotes como forma de tornar mais legível e usual o seu desenvolvimento, entre os pacotes encontram : Despachante, Memória, Processo e Recurso.

 
\subsection{Dificuldades}
Entre as dificuldades encontradas durante a implementação do projeto, destacam-se a implementação do algoritmo de escalonamento Round Robin e a técnica de paginação da memória em páginas.

Para a implementação do algoritmo de escalonamento foi necessário uma revisão de literatura e uma sequência de aulas do Algoritmos de Escalonamento da UNIVESP e Neso Academy ambas disponível na Plataforma YouTube.

Já para a implementação da técnica de Paginação de Memória, foi preciso reassistir a aula sobre o assunto disponível na Plataforma Teams.


\subsection{Testes}
Para realização de testes durante a implementação foi necessário o desenvolvimento de um pacote especifico dentro do projeto para controle e verificação de eficiência de algumas funções específicas do projeto, como disparo de processo, carregamento de tarefas de arquivo externo, impressão de debugs  e outros.


\section{Resultados}
O pseudo sistema operacional consegue disparar processos em modo Kernel e modo Usuário, sendo que no modo Kernel o processo executa pela técnica de First Fit em tempo real após fila de espera e no modo usuário pelo escalonamento Round robin com 3 filas de prioridade com quantum de 1 segundo, no caso do VLOS 1 segundo é igual a 10 ciclos de processamento.


\section{Considera\c c\~oes finais}
% TEXTO DA CONCLUSAO
Foi um trabalho complexo e que requereu muito tempo e dedicação para seu planejamento, desenvolvimento e teste.

\renewcommand{\bibsection}{\section*{REFER\^ENCIAS BIBLIOGR\'AFICAS}}
\bibliographystyle{abntex2-alf}
\bibliography{Referencias}
\end{document}